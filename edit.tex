
%%%%%%%%%%%%%%%%%%%%%%%%%%%%%%%%%%%%%%%%%%%%%%%%%%%%%%%%%%%%%%%%%%%%%
%% This is a (brief) model paper using the achemso class
%% The document class accepts keyval options, which should include
%% the target journal and optionally the manuscript type.
%%%%%%%%%%%%%%%%%%%%%%%%%%%%%%%%%%%%%%%%%%%%%%%%%%%%%%%%%%%%%%%%%%%%%
\documentclass[journal=jceaax,manuscript=article]{achemso}
\usepackage{amssymb}
\usepackage{amsmath}
\usepackage{comment}
%%%%%%%%%%%%%%%%%%%%%%%%%%%%%%%%%%%%%%%%%%%%%%%%%%%%%%%%%%%%%%%%%%%%%
%% Place any additional packages needed here.  Only include packages
%% which are essential, to avoid problems later.
%%%%%%%%%%%%%%%%%%%%%%%%%%%%%%%%%%%%%%%%%%%%%%%%%%%%%%%%%%%%%%%%%%%%%
\usepackage{multirow}
\usepackage{chemformula} % Formula subscripts using \ch{}
\usepackage[T1]{fontenc} % Use modern font encodings
%%%%%%%%%%%%%%%%%%%%%%%%%%%%%%%%%%%%%%%%%%%%%%%%%%%%%%%%%%%%%%%%%%%%%
%% If issues arise when submitting your manuscript, you may want to
%% un-comment the next line.  This provides information on the
%% version of every file you have used.
%%%%%%%%%%%%%%%%%%%%%%%%%%%%%%%%%%%%%%%%%%%%%%%%%%%%%%%%%%%%%%%%%%%%%
%%\listfiles

%%%%%%%%%%%%%%%%%%%%%%%%%%%%%%%%%%%%%%%%%%%%%%%%%%%%%%%%%%%%%%%%%%%%%
%% Place any additional macros here.  Please use \newcommand* where
%% possible, and avoid layout-changing macros (which are not used
%% when typesetting).
%%%%%%%%%%%%%%%%%%%%%%%%%%%%%%%%%%%%%%%%%%%%%%%%%%%%%%%%%%%%%%%%%%%%%
\newcommand*\mycommand[1]{\texttt{\emph{#1}}}

\setkeys{acs}{maxauthors=10}
\setkeys{acs}{etalmode=truncate}

\usepackage{float} 
\usepackage{tikz}
\usepackage{multirow}
\usepackage{longtable}
%%\usepackage{bm}
\usepackage{caption}
\usepackage{arydshln}
\usepackage{xr} % for cross reference



%%%%%%%%%%%%%%%%%%%%%%%%%%%% Barnabas' Added packages %%%%%%%%%%%%%
\usepackage{verbatim} %for easy writing of multi-line comments
\usepackage{csvsimple}
\usepackage{tabularx}
\usepackage{array}
%%%\usepackage{array}
%%%%%%%%%%%%%%%%%%%%%%%%%%%%


% this is for cross reference 
\makeatletter
\newcommand*{\addFileDependency}[1]{% argument=file name and extension
  \typeout{(#1)}
  \@addtofilelist{#1}
  \IfFileExists{#1}{}{\typeout{No file #1.}}
}
\makeatother

\newcommand*{\myexternaldocument}[1]{%
    \externaldocument{#1}%
    \addFileDependency{#1.tex}%
    \addFileDependency{#1.aux}%
}
\myexternaldocument{SI}

%%%%%%%%%%%%%%%%%%%%%%%%%%%%%%%%%%%%%%%%%%%%%%%%%%%%%%%%%%%%%%%%%%%%%
%% Meta-data block
%% ---------------
%% Each author should be given as a separate \author command.
%%
%% Corresponding authors should have an e-mail given after the author
%% name as an \email command. Phone and fax numbers can be given
%% using \phone and \fax, respectively; this information is optional.
%%
%% The affiliation of authors is given after the authors; each
%% \affiliation command applies to all preceding authors not already
%% assigned an affiliation.
%%
%% The affiliation takes an option argument for the short name.  This
%% will typically be something like "University of Somewhere".
%%
%% The \altaffiliation macro should be used for new address, etc.
%% On the other hand, \alsoaffiliation is used on a per author basis
%% when authors are associated with multiple institutions.
%%%%%%%%%%%%%%%%%%%%%%%%%%%%%%%%%%%%%%%%%%%%%%%%%%%%%%%%%%%%%%%%%%%%%
\usepackage[symbol]{footmisc}
\author{Barnabas P. Agbodekhe}
\author{Dinis O. Abranches}
\author{Montana N. Carlozo}
\author{Kyla D. Jones}
\author{Alexander W.~Dowling}
\author{Edward J. Maginn}
\email{ed@nd.edu}
%\phone{+123 (0)123 4445556}
%\fax{+123 (0)123 4445557}
\affiliation[University of Notre Dame]
{Department of Chemical and Biomolecular Engineering, University of Notre Dame, Notre Dame, IN 46556, USA}
%\alsoaffiliation[Second University]{Department of Chemistry, Second University, Nearby Town}


%%%%%%%%%%%%%%%%%%%%%%%%%%%%%%%%%%%%%%%%%%%%%%%%%%%%%%%%%%%%%%%%%%%%%
%% The document title should be given as usual. Some journals require
%% a running title from the author: this should be supplied as an
%% optional argument to \title.
%%%%%%%%%%%%%%%%%%%%%%%%%%%%%%%%%%%%%%%%%%%%%%%%%%%%%%%%%%%%%%%%%%%%%
\title[An \textsf{achemso} demo]
  {Integrating Group Contribution Models with Gaussian Process Regression for Simple, Generalizable, and Accurate Thermophysical Property Prediction}

%%%%%%%%%%%%%%%%%%%%%%%%%%%%%%%%%%%%%%%%%%%%%%%%%%%%%%%%%%%%%%%%%%%%%
%% Some journals require a list of abbreviations or keywords to be
%% supplied. These should be set up here, and will be printed after
%% the title and author information, if needed.
%%%%%%%%%%%%%%%%%%%%%%%%%%%%%%%%%%%%%%%%%%%%%%%%%%%%%%%%%%%%%%%%%%%%%
\abbreviations{ GC}
\keywords{Group Contribution, thermophysical properties}

%%%%%%%%%%%%%%%%%%%%%%%%%%%%%%%%%%%%%%%%%%%%%%%%%%%%%%%%%%%%%%%%%%%%%
%% The manuscript does not need to include \maketitle, which is
%% executed automatically.
%%%%%%%%%%%%%%%%%%%%%%%%%%%%%%%%%%%%%%%%%%%%%%%%%%%%%%%%%%%%%%%%%%%%%
\begin{document}

\sloppy  % stops long words from running over the margin
%%%%%%%%%%%%%%%%%%%%%%%%%%%%%%%%%%%%%%%%%%%%%%%%%%%%%%%%%%%%%%%%%%%%%
%% The "tocentry" environment can be used to create an entry for the
%% graphical table of contents. It is given here as some journals
%% require that it is printed as part of the abstract page. It will
%% be automatically moved as appropriate.
%%%%%%%%%%%%%%%%%%%%%%%%%%%%%%%%%%%%%%%%%%%%%%%%%%%%%%%%%%%%%%%%%%%%%

%%%%%%%%%%%%%%%%%% TO BE EDITED %%%%%%%%%%%%%%%%%%%%
%\begin{tocentry}
%\begin{figure}[H]
%    \centering
%    \includegraphics[width=8cm,scale=0.5]{TOC.png}
%    %\caption{}
%    \label{fig:toc}
%\end{figure}
%Alchemical transformation in conjunction with Hamiltonian replica exchange %molecular dynamics to estimate the full solubility isotherm of %hydrofluorocarbon in ionic liquids using phase equilibrium theory.
%\end{tocentry}

%%%%%%%%%%%%%%%%%%%%%%%%%%%%%%%%%%%%%%%%%%%%%%%%%%%%%%%%%%%%%%%%%%%%%
%% The abstract environment will automatically gobble the contents
%% if an abstract is not used by the target journal.
%%%%%%%%%%%%%%%%%%%%%%%%%%%%%%%%%%%%%%%%%%%%%%%%%%%%%%%%%%%%%%%%%%%%%
\begin{abstract}

\end{abstract}


%%%%%%%%%%%%%%%%%%%%%%%%%%%%%%%%%%%%%%%%%%%%%%%%%%%%%%%%%%%%%%%%%%%%%
%% Start the main part of the manuscript here.
%%%%%%%%%%%%%%%%%%%%%%%%%%%%%%%%%%%%%%%%%%%%%%%%%%%%%%%%%%%%%%%%%%%%%

\section{Introduction}


\subsection{FFs Ranking}


\begin{table}[H]
\centering
\begin{tabular}{p{2.9cm}p{1.8cm}p{1.6cm}p{1.5cm}p{1.6cm}p{1.6cm}p{1.6cm}p{1.6cm}} 
Substance Name&  Molecular Formula&  CAS No.&  MW /$gmol^{-1}$&  JR GC $H_{vap}$ /$kJmol^{-1}$&  Exp. $H_{vap}$ /$kJmol^{-1}$&  GCGP $H_{vap}$ /$kJmol^{-1}$& GCGP Pred. Unc. /$kJmol^{-1}$ 

%\hline 

\\
         perfluorodecalin&  $C_{10}F_{18}$&  306-94-5&  462.08&  9.68&  35.80&  35.77& 12.45
\\
         perfluorononane&  $C_{9}F_{20}$&  375-96-2&  488.07&  7.62&  34.22&  35.84& 15.09
\\
         perfluoro-2-methylpentane&  $C_{6}F_{14}$&  355-04-4&  338.05&  9.74&  27.89&  25.73& 16.37
\\
         pentadecafluoro-octanoic acid&  $C_{8}HF_{15}O_{2}$&  335-67-1&  414.07&  29.24&  40.40&  41.34& 18.08
\\
         heptadecafluoro-nonanoic acid&  $C_{9}HF_{17}O_{2}$&  375-95-1&  464.08&  28.54&  43.03&  45.15& 19.30
\\
\end{tabular}
\caption{Caption}
\label{tab:my_label}
\end{table}



\begin{table}[H]
    \centering
    \begin{tabular}{>{\centering\arraybackslash}p{1.0cm}>{\centering\arraybackslash}p{1.0cm}>{\centering\arraybackslash}p{0.75cm}>{\centering\arraybackslash}p{0.75cm}>{\centering\arraybackslash}p{1.0cm}>{\centering\arraybackslash}p{1.0cm}>{\centering\arraybackslash}p{0.75cm}>{\centering\arraybackslash}p{0.75cm}>{\centering\arraybackslash}p{1cm}>{\centering\arraybackslash}p{0.75cm}>{\centering\arraybackslash}p{0.75cm}>{\centering\arraybackslash}p{0.75cm}>{\centering\arraybackslash}p{1cm}>{\centering\arraybackslash}p{0.75cm}}
       \vspace{1.15cm}  kern ID& 
       \vspace{1.15cm} Cond   N& 
       \vspace{0.66cm} R2  train     1&  
 \vspace{0.66cm} R2  test   1& 
 \vspace{0.146cm} (\%) MAPD test \hspace{0.5cm}  1&  
 ($kJ \cdot mol^{-1}$) MAE test  \hspace{0.5cm}   1&  \vspace{0.66cm} R2 train   4& 
 \vspace{0.66cm} R2 test   4& 
 \vspace{0.146cm} (\%) MAPD test  \hspace{0.5cm}  4&  
 ($kJ \cdot mol^{-1}$) MAE test  \hspace{0.5cm}   4& \vspace{0.66cm} R2   train      5& 
 \vspace{0.66cm} R2   test     5& 
 \vspace{0.146cm} (\%) MAPD test  \hspace{0.5cm}   5&
 ($kJ \cdot mol^{-1}$) MAE test  \hspace{0.5cm}   5
 
%\\ 

\\
          rbf(2)   &  221.3&  0.86&  0.9&  5.14&  1.75&  0.86&  0.91&  5.1&  1.73& 0.86& 0.91& 5.13&1.74
\\
         rq(2)   &  244.34&  0.86&  0.9&  5.15&  1.76&  0.86&  0.9&  5.11&  1.74& 0.86& 0.91& 5.13&1.74
\\
         mt1(2) &  155&  0.9&  0.89&  5.53&  1.9&  0.9&  0.89&  5.57&  1.91& 0.89& 0.9& 5.49&1.88
\\
         mt3(2)&  193.68&  0.86&  0.9&  5.21&  1.79&  0.86&  0.9&  5.18&  1.77& 0.86& 0.9& 5.19&1.77
\\
         mt5(2) &  204.09&  0.86&  0.9&  5.18&  1.77&  0.86&  0.9&  5.14&  1.75& 0.86& 0.9& 5.16&1.76
\\
         rbf    &  221.3&  0.85&  0.88&  5.23&  1.84&  0.85&  0.88&  5.11&  1.8& 0.86& 0.88& 5.36&1.89
\\
         rq   &  244.34&  0.86&  0.88&  5.34&  1.87&  0.86&  0.88&  5.28&  1.86& 0.86& 0.88& 5.34&1.88
\\
         mt1  &  155&  0.91&  0.87&  5.42&  1.9&  0.91&  0.87&  5.39&  1.89& 0.9& 0.87& 5.36&1.88
\\
         mt3  &  193.68&  0.86&  0.88&  5.29&  1.85&  0.86&  0.88&  5.23&  1.84& 0.86& 0.88& 5.29&1.86
\\
 mt5  & 204.09& 0.86& 0.88& 5.28& 1.85& 0.85& 0.88& 5.18& 1.83& 0.86& 0.88& 5.31&1.87
\\
    \end{tabular}
    \caption{Kernel Sweep Summary for $\Delta H_{vap}$ using models 1, 4, and 5}
    \label{tab:hvap_ksweep}
\end{table}






\end{document}



\section{Conclusion}

%%%%%%%%%%%%%%%%%%%%%%%%%%%%%%%%%%%%%%%%%%%%%%%%%%%%%%%%%%%%%%%%%%%%%
%% The "Acknowledgement" section can be given in all manuscript
%% classes.  This should be given within the "acknowledgement"
%% environment, which will make the correct section or running title.
%%%%%%%%%%%%%%%%%%%%%%%%%%%%%%%%%%%%%%%%%%%%%%%%%%%%%%%%%%%%%%%%%%%%%
\begin{acknowledgement}
The authors thank the financial support provided by the National Science Foundation, EFRI DChem: Next-generation Low Global Warming Refrigerants, Award no. 2029354 and the computing resources from the Center for Research Computing (CRC) at the University of Notre Dame. 
\end{acknowledgement}

%%%%%%%%%%%%%%%%%%%%%%%%%%%%%%%%%%%%%%%%%%%%%%%%%%%%%%%%%%%%%%%%%%%%%
%% The same is true for Supporting Information, which should use the
%% suppinfo environment.
%%%%%%%%%%%%%%%%%%%%%%%%%%%%%%%%%%%%%%%%%%%%%%%%%%%%%%%%%%%%%%%%%%%%%
\begin{suppinfo}
The Supporting Information is available free of charge at ...\\


The code of this workflow is available to the public at\\ https://github.com/MaginnGroup/\\


\end{suppinfo}

%%%%%%%%%%%%%%%%%%%%%%%%%%%%%%%%%%%%%%%%%%%%%%%%%%%%%%%%%%%%%%%%%%%%%
%% The appropriate \bibliography command should be placed here.
%% Notice that the class file automatically sets \bibliographystyle
%% and also names the section correctly.
%%%%%%%%%%%%%%%%%%%%%%%%%%%%%%%%%%%%%%%%%%%%%%%%%%%%%%%%%%%%%%%%%%%%%
%%%\bibliography{FF_Testing_refs}

%\bibliography{FF_ref_check}

\end{document}
